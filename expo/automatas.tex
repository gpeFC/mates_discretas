
% Este documento LaTeX fue dise�ado por profesores  del Departamento de Matem�ticas 
% de la Universidad de Antioqua (http://ciencias.udea.edu.co/). Usted puede modificarlo
% y personalizarlo a su gusto bajo los t�rminos de la licencia de documentaci�n libre GNU.
% http://es.wikipedia.org/w/index.php?title=Licencia_de_documentaci%C3%B3n_libre_de_GNU&oldid=15717448

%\documentclass[serif,9pt]{beamer}
%\setbeamertemplate{navigation symbols}{}

%\setbeamercolor{frametitle}{fg=black,bg=white}
%\setbeamercolor{title}{fg=black,bg=yellow!85!orange}
%\usetheme{AnnArbor}

%\documentclass{beamer}

\documentclass[Arial,11pt]{beamer}
\usetheme{Dresden}
%\usepackage{german}
\usepackage[latin1]{inputenc}
\usepackage[spanish]{babel}


\begin{document}

\title{\textbf{Aut\'omatas}} 
\author{Emanuel Garc\'ia P\'erez}
\date{\today}
\institute[]{%
  Facultad de Ciencias\\
  Universidad Aut�noma del Estado de Morelos}
\logo{\includegraphics[scale=0.2]{log}}



\begin{frame}
\titlepage
\end{frame}

\begin{frame}
\transdissolve[duration=0.2]
\frametitle{Contenido}
\tableofcontents
\end{frame} 


\section{Aut\'omatas} 

\subsection{Aut\'omata de Estado Finito (Determinista)}
\frame{
\transdissolve[duration=0.2]
\frametitle{�Qu\'e es un aut\'omata de estado finito (determinista)?}
}

\frame{
\transdissolve[duration=0.2]
\frametitle{Definici\'on formal: AFD}
}


\subsection{Aut\'omata de Estado Finito no Determinista}
\frame{
\transdissolve[duration=0.2]
\frametitle{�Qu\'e es un aut\'omata de estado finito no determinista?}
}

\frame{
\transdissolve[duration=0.2]
\frametitle{Definici\'on formal: AFND}
}


\section{Referencias} 

\begin{frame}
\transdissolve[duration=0.2]
\frametitle<presentation>{Referencias}


\begin{thebibliography}{10}

\bibitem[Devlin, 2002]{Devlin}
A.J.~Chorin, J.E.~Marsden.
\newblock {\em A Mathematical Introduction to Fluid Mechanics}
\newblock Springer-Verlag, 1980.

\bibitem[Devlin, 2002]{Devlin}
K.~Devlin.
\newblock {\em The Millenium Problems. The Seven Greatest Unsolved Mathematical Puzzles of Our Time}
\newblock Basic Books, 2002.

\bibitem{Fefferman} 
C.~Fefferman.
\newblock {\em Clay Mathematics Institute, Millenium Problems. Official problem description}.
\newblock 
\href{http://www.claymath.org/millennium/Navier-Stokes\_Equations/}{http://www.claymath.org/millennium/Navier-Stokes\_Equation}

\bibitem[Wikipedia contributors, 200]{Wikipedia}
Wikipedia contributors
\newblock {\em Navier-Stokes equations}
\newblock Wikipedia, The Free Encyclopedia., 2008.
\newblock 
\href{http://en.wikipedia.org/wiki/Navier-Stokes\_equations}{http://en.wikipedia.org/wiki/Navier-Stokes\_equations}

\end{thebibliography}
\end{frame}


\end{document}